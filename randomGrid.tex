%% This is file `elsarticle-template-1-num.tex',
%%
%% Copyright 2009 Elsevier Ltd
%%
%% This file is part of the 'Elsarticle Bundle'.
%% ---------------------------------------------
%%
%% It may be distributed under the conditions of the LaTeX Project Public
%% License, either version 1.2 of this license or (at your option) any
%% later version.  The latest version of this license is in
%%    http://www.latex-project.org/lppl.txt
%% and version 1.2 or later is part of all distributions of LaTeX
%% version 1999/12/01 or later.
%%
%% Template article for Elsevier's document class `elsarticle'
%% with numbered style bibliographic references
%%
%% $Id: elsarticle-template-1-num.tex 149 2009-10-08 05:01:15Z rishi $
%% $URL: http://lenova.river-valley.com/svn/elsbst/trunk/elsarticle-template-1-num.tex $
%%
\documentclass[preprint,12pt]{elsarticle}

%% Use the option review to obtain double line spacing
%% \documentclass[preprint,review,12pt]{elsarticle}

%% Use the options 1p,twocolumn; 3p; 3p,twocolumn; 5p; or 5p,twocolumn
%% for a journal layout:
%% \documentclass[final,1p,times]{elsarticle}
%% \documentclass[final,1p,times,twocolumn]{elsarticle}
%% \documentclass[final,3p,times]{elsarticle}
%% \documentclass[final,3p,times,twocolumn]{elsarticle}
%% \documentclass[final,5p,times]{elsarticle}
%% \documentclass[final,5p,times,twocolumn]{elsarticle}

%% The graphicx package provides the includegraphics command.
\usepackage{graphicx}
%% The amssymb package provides various useful mathematical symbols
\usepackage{amssymb}
%% The amsthm package provides extended theorem environments
%% \usepackage{amsthm}

%% The lineno packages adds line numbers. Start line numbering with
%% \begin{linenumbers}, end it with \end{linenumbers}. Or switch it on
%% for the whole article with \linenumbers after \end{frontmatter}.
\usepackage{lineno}

%% natbib.sty is loaded by default. However, natbib options can be
%% provided with \biboptions{...} command. Following options are
%% valid:

%%   round  -  round parentheses are used (default)
%%   square -  square brackets are used   [option]
%%   curly  -  curly braces are used      {option}
%%   angle  -  angle brackets are used    <option>
%%   semicolon  -  multiple citations separated by semi-colon
%%   colon  - same as semicolon, an earlier confusion
%%   comma  -  separated by comma
%%   numbers-  selects numerical citations
%%   super  -  numerical citations as superscripts
%%   sort   -  sorts multiple citations according to order in ref. list
%%   sort&compress   -  like sort, but also compresses numerical citations
%%   compress - compresses without sorting
%%
%% \biboptions{comma,round}

% \biboptions{}

\journal{Journal Name}

\begin{document}

\begin{frontmatter}

%% Title, authors and addresses

\title{Unnecessarily Complicated Research Title}

%% use the tnoteref command within \title for footnotes;
%% use the tnotetext command for the associated footnote;
%% use the fnref command within \author or \address for footnotes;
%% use the fntext command for the associated footnote;
%% use the corref command within \author for corresponding author footnotes;
%% use the cortext command for the associated footnote;
%% use the ead command for the email address,
%% and the form \ead[url] for the home page:
%%
%% \title{Title\tnoteref{label1}}
%% \tnotetext[label1]{}
%% \author{Name\corref{cor1}\fnref{label2}}
%% \ead{email address}
%% \ead[url]{home page}
%% \fntext[label2]{}
%% \cortext[cor1]{}
%% \address{Address\fnref{label3}}
%% \fntext[label3]{}


%% use optional labels to link authors explicitly to addresses:
%% \author[label1,label2]{<author name>}
%% \address[label1]{<address>}
%% \address[label2]{<address>}

\author{John Smith}

\address{California, United States}

\begin{abstract}
%% Text of abstract
Suspendisse potenti. Suspendisse quis sem elit, et mattis nisl. Phasellus consequat erat eu velit rhoncus non pharetra neque auctor. Phasellus eu lacus quam. Ut ipsum dolor, euismod aliquam congue sed, lobortis et orci. Mauris eget velit id arcu ultricies auctor in eget dolor. Pellentesque suscipit adipiscing sem, imperdiet laoreet dolor elementum ut. Mauris condimentum est sed velit lacinia placerat. Vestibulum ante ipsum primis in faucibus orci luctus et ultrices posuere cubilia Curae; Nullam diam metus, pharetra vitae euismod sed, placerat ultrices eros. Aliquam tincidunt dapibus venenatis. In interdum tellus nec justo accumsan aliquam. Nulla sit amet massa augue.
\end{abstract}

\begin{keyword}
Science \sep Publication \sep Complicated
%% keywords here, in the form: keyword \sep keyword

%% MSC codes here, in the form: \MSC code \sep code
%% or \MSC[2008] code \sep code (2000 is the default)

\end{keyword}

\end{frontmatter}

%%
%% Start line numbering here if you want
%%
\linenumbers

%% main text
\section{Introduction}
\label{S:1}

There is a growing need for the development of innovative approaches so that management of all marine stocks not just those of high commercial value can be included into the Common Fisheries Policy (CFP) framework. Management objectives under the CFP are to recover stocks and to maintain stocks within safe biological limits, including by-catch species. These conservation measures can include biological target reference points e.g. and/or population size (stock biomass), yields (catches), long–term yields and fishing mortality against which the preservation of stocks within such limits are assessed. These are often referred to as management procedures or harvesting strategies which include the operational component harvest control rule (HCR) based on indicators (e.g. monitoring data or models) of stock status. To test the performance of candidate management strategies often requires evaluation of alternative hypothesis about the dynamics of the system e.g. population dynamics (e.g. life history dynamics) and the behaviour of the fishery (e.g range contraction and density dependence) etc..  Due to the nature of conflicting objectives, stakeholder interests and the uncertainty in the dynamics of the resource and/ or the plausibility of alternative hypotheses can lead to poor decision making and can be problematic when defining management policy.

An intense area of work being researched over the last 2 decades is Management Strategy Evaluation (MSE), which focuses on the broader aspects of fishing (the Ecosystem) whereby different management options are tested against a range of objectives (see Kell et al., 2007).  The approach is not to come up with a definitive answer, but to lay-bare the trade offs associated with each management objective, along with identifying and incorporating uncertainties in the evaluation and communicating the results effectively to client groups and decision-makers. MSE is not intended to be complex but to provide a robust framework that account for conflicting poorly defined objectives and uncertainties that have been absent in conventional management (Kell et al., 2007).  
To better understand the performance of a range of management procedures we aim to test generic empirical HCR (based on catch per unit effort – CPUE indices) that maximises yield without stock collapse for ICES data-limited fisheries. 

Often empirical harvest control rules require extensive exhaustive parameter searches to tune hyper-parameters that aren’t directly learnt from estimators.  This requires a technique known as a grid search that extensively searches for all combinations of all parameters. In contrast and some what less time consuming, other efficient parameter search strategy’s can be considered given range of parameter space and a known distribution a sample can be obtained and is known as a random search.  

To test case specific harvest strategies (via simulation) within the MSE, we will implement a management procedure based on a empirical model that adjusts yield depending on stock status for a given range set of hype-parameters for the harvest strategy and test their robustness to uncertainty.  This approach could also help identify similar conditions across species where particular advice rules are likely to work well, and where they perform poorly for a given a set of hyper-parameters. Assessment is made as to the performance of each HCR via a set of utilities: safety (a proportion B/BMSY >1), yield (a proportion- $yield/MSY$), kobe proportion (proportion of years that stay in the green zone of kobe plot ($B/B_{MSY} >1$), and Yield Annual Variation (yield changes by 10\% year on year).  

\begin{itemize}
\item Bullet point one
\item Bullet point two
\end{itemize}


\section{Material and Methods}
\subsection{Materials}
\subsection{Methods}
\subsubsection{Operating Model}
\subsubsection{Management Procedure}

\subsubsection{Random Search}


\section{Results}
\section{Discussion}
\section{Conclusions}


\bibliographystyle{model1-num-names}
\bibliography{/home/laurence/Desktop/flr/equations/tex/refs.bib} 

%% Authors are advised to submit their bibtex database files. They are
%% requested to list a bibtex style file in the manuscript if they do
%% not want to use model1-num-names.bst.

%% References without bibTeX database:

% \begin{thebibliography}{00}

%% \bibitem must have the following form:
%%   \bibitem{key}...
%%

% \bibitem{}

% \end{thebibliography}


\end{document}

%%
%% End of file `elsarticle-template-1-num.tex'.